\documentclass{article}

% if you need to pass options to natbib, use, e.g.:
% \PassOptionsToPackage{numbers, compress}{natbib}
% before loading nips_2017
%
% to avoid loading the natbib package, add option nonatbib:
% \usepackage[nonatbib]{nips_2017}

%\usepackage{main}

% to compile a camera-ready version, add the [final] option, e.g.:
\usepackage[final]{main}

\usepackage[utf8]{inputenc} % allow utf-8 input
\usepackage[T1]{fontenc}    % use 8-bit T1 fonts
\usepackage{hyperref}       % hyperlinks
\usepackage{url}            % simple URL typesetting
\usepackage{booktabs}       % professional-quality tables
\usepackage{amsfonts}       % blackboard math symbols
\usepackage{nicefrac}       % compact symbols for 1/2, etc.
\usepackage{microtype}      % microtypography
\usepackage{multicol}
\usepackage{graphicx}
\usepackage{amsmath}
\usepackage{bbm}
\usepackage{enumerate}
\usepackage[linguistics]{forest}
\usepackage{adjustbox}
\usepackage{bbm}
\usepackage{stmaryrd}
\usepackage{tikz}
%\usepackage[margin=0.5in]{geometry}
%\DeclareMathOperator*{\argmax}{argmax}

\usepackage{listings}
\lstset{
  basicstyle=\ttfamily,
  mathescape
}
\usepackage{color}
 
\definecolor{codegreen}{rgb}{0,0.6,0}
\definecolor{codegray}{rgb}{0.5,0.5,0.5}
\definecolor{codepurple}{rgb}{0.58,0,0.82}
\definecolor{backcolour}{rgb}{0.95,0.95,0.92}
 
\lstdefinestyle{mystyle}{
    backgroundcolor=\color{backcolour},   
    commentstyle=\color{codegreen},
    keywordstyle=\color{magenta},
    numberstyle=\ttfamily\tiny\color{codegray},
    stringstyle=\color{codepurple},
    basicstyle=\ttfamily\small,
    columns=fullflexible,
    breakatwhitespace=false,         
    breaklines=true,                 
    captionpos=t,                    
    keepspaces=true,                 
    numbers=left,                    
    numbersep=5pt,                  
    showspaces=false,                
    showstringspaces=false,
    showtabs=false,                  
    tabsize=4,
}
 
\lstset{style=mystyle}

\title{Fundamental Algorithms - Spring 2018\\
       \Large Homework 11}
\graphicspath{{images/}}
\setcitestyle{round, sort, numbers}

% The \author macro works with any number of authors. There are two
% commands used to separate the names and addresses of multiple
% authors: \And and \AND.
%
% Using \And between authors leaves it to LaTeX to determine where to
% break the lines. Using \AND forces a line break at that point. So,
% if LaTeX puts 3 of 4 authors names on the first line, and the last
% on the second line, try using \AND instead of \And before the third
% author name.

\author{
  Daniel Rivera Ruiz\\
  Department of Computer Science\\
  New York University\\
  \href{mailto:drr342@nyu.edu}{\texttt{drr342@nyu.edu}}\\
}

\begin{document}

\maketitle

% \cite{} - in-line citation author, year in parenthesis.
% \citep{} - all citation info in parenthesis.

%	\begin{figure}[ht]
%		\centering
%		\frame{
%            \includegraphics[width=1.0\linewidth]{tree.png}
%       }
%		\caption{Classification results for the sentence \textit{"There are slow and repetitive parts, but it has just enough spice to keep it                  interesting."} using the Stanford Sentiment Treebank. As can be seen, sentiment scores are available for each phrase.}
%		\label{tree}
%	\end{figure}

\begin{enumerate}[1.]

    \item Since the graph is complete, all the vertices are in the adjacency list of the root $r = 1$ and therefore will be added to the priority queue $Q$ during the initialization with $\pi(j) = 1$ and $k(j) = (j - 1)^3$ for $2 \leq j \leq n$.\\
    At the first iteration of the algorithm, the minimal element in $Q$ is $2$ because $k(j)$ is a strictly increasing function\footnotemark. After removing $2$ from $Q$ and adding it to $S$, we have to update $\pi$ and $k$ for all the values in $Q$ because they are all in $adj(2)$ and they are all closer to $2$ than they are to $1$. Therefore, we have to make $\pi(j) = 2$ and $k(j) = (j - 2)^3$ for $3 \leq j \leq n$.\\
    Following this intuition, at the $i^{th}$ iteration of the algorithm the minimal element in $Q$ will be $i$ and after extracting it $Q$ must be updated as follows: $\pi(j) = i$ and $k(j) = (j - i)^3$ for $i + 1 \leq j \leq n$.
    \begin{enumerate}[(a)]
        \item Under these conditions for a graph with $n$ vertices where $n = 500$, the first 211 elements inserted in the MST have to be $\{1, 2, 3, \ldots , 211\}$.
        \item After inserting the $211^{th}$ element and updating $\pi$ and $k$ for the remaining elements in $Q$ we will have $\pi(309) = 211$ and $k(309) = (309 - 211)^3 = 98^3 = 941192$.
    \end{enumerate}
    \footnotetext{In the definition of $k$ we assume $j > i$ in order to assure $(j - i)^3 > 0$. This is a necessary condition since the definition of the MST problem requires that all weights $w$ be greater or equal than 0.}
    
    \item The following table shows the evolution of the \texttt{Extended-Euclid} algorithm to calculate $d$, $x$ and $y$. Due to the recursion, the columns $a$, $b$, $q$ and $r$ are calculated from the top down, and afterwards $d$, $x$ and $y$ are calculated from the bottom up.
	\begin{table}[ht]
		\centering
		\begin{tabular}{ccccccc}
			\toprule
			$a$ & $b$ & $q$ & $r$ & $d$ & $x$ & $y$ \\
			\midrule
            144 & 89 & 1 & 55 & 1 & 34 & -55 \\
            89 & 55 & 1 & 34 & 1 & -21 & 34 \\
            55 & 34 & 1 & 21 & 1 & 13 & -21 \\
            34 & 21 & 1 & 13 & 1 & -8 & 13 \\
            21 & 13 & 1 & 8 & 1 & 5 & -8 \\
            13 & 8 & 1 & 5 & 1 & -3 & 5 \\
            8 & 5 & 1 & 3 & 1 & 2 & -3 \\
            5 & 3 & 1 & 2 & 1 & -1 & 2 \\
            3 & 2 & 1 & 1 & 1 & 1 & -1 \\
            2 & 1 & 2 & 0 & 1 & 0 & 1 \\
            1 & 0 & - & - & 1 & 1 & 0 \\
			\bottomrule
		\end{tabular}
    \end{table}
    
    At the end of the algorithm we get $d = 1$, $x = 34$ and $y = -55$, which satisfies $d = ax + by \Rightarrow 1 = (144)(34) + (89)(-55)$.
    
    \item To solve $\frac{311}{507}$ in $\mathbb{Z}_{1000}$ we observe that $gcd(507, 1000) = 1$ and therefore $507$ has a multiplicative inverse in $\mathbb{Z}_{1000}$. The problem reduces to finding $y \equiv 507^{-1} (mod\,1000)$ and calculating $311y\,(mod\,1000)$.\\
    To find $y$, we use \texttt{Extended-Euclid} with $a = 1000$ and $b = 507$:
	\begin{table}[ht]
		\centering
		\begin{tabular}{ccccccc}
			\toprule
			$a$ & $b$ & $q$ & $r$ & $d$ & $x$ & $y$ \\
			\midrule
			1000 & 507 & 1 & 493 & 1 & 181 & -357 \\
			507 & 493 & 1 & 14 & 1 & -176 & 181 \\
			493 & 14 & 35 & 3 & 1 & 5 & -176 \\
            14 & 3 & 4 & 2 & 1 & -1 & 5 \\
            3 & 2 & 1 & 1 & 1 & 1 & -1 \\
            2 & 1 & 2 & 0 & 1 & 0 & 1 \\
            1 & 0 & - & - & 1 & 1 & 0 \\
			\bottomrule
		\end{tabular}
    \end{table}
    \newpage
    So we get $1 = (1000)(181) + (507)(-357)$ and therefore $y = -357$ in $\mathbb{Z}_{1000}$. Finally we calculate $(311)(-357) \equiv -111027 \equiv -27\,(mod\,1000)$:
    \begin{equation*}
        \frac{311}{507} = -27 \quad \text{in} \ \mathbb{Z}_{1000}
    \end{equation*}
    
    \item We observe that $101$ and $103$ are both prime numbers and therefore $gcd(101, 103) = 1$. Under this condition, we can use the \emph{Chinese Reminder Theorem} to solve the system:
    \begin{enumerate}[(a)]
        \item First we need to calculate $y \equiv (101^{-1})(59 - 34) \equiv (101^{-1})(25)\,(mod\,103)$.
        \item To find $101^{-1}$ in $\mathbb{Z}_{103}$ we use \texttt{Extended-Euclid} with $a = 103$ and $b = 101$, which returns $d = 1$, $x = -50$ and $y = 51$, so $101^{-1} = 51$ in $\mathbb{Z}_{103}$.
        \item We get the value of $y \equiv (51)(25) \equiv 1275 \equiv 39\,(mod\,103)$.
        \item Finally, we replace the value of $y$ in $x = 34 + 101y = 34 + (101)(39) = 3973$.
    \end{enumerate}
    The solution to the original system is given by
    \begin{equation*}
        x \equiv 3973 \, (mod \, 101 \cdot 103) \Rightarrow x \equiv 3973 \, (mod \, 10403)
    \end{equation*}
    
    \item To compute $2^{1072}$ in $\mathbb{Z}_{1073}$ we follow the modular exponentiation algorithm:
    \begin{enumerate}[(a)]
        \item Write $B = 1072$ in binary form: $1072 = 1024 + 32 + 16 = 2^{10} + 2^5 + 2^4$.
        \item Repeatedly square $A = 2$ in $\mathbb{Z}_{1073}$:
    	\begin{table}[ht]
    		\centering
    		\begin{tabular}{cccc}
    			\toprule
    			$i$ & $2^i$ & $2^{2^i}$ & $2^{2^i}$ in $\mathbb{Z}_{1073}$ \\
    			\midrule
    			0 & 1 & 2 & 2 \\
    			1 & 2 & 4 & 4 \\
    			2 & 4 & 16 & 16 \\
    			3 & 8 & 256 & 256 \\
    			4 & 16 & 65536 & 83 \\
    			5 & 32 & 6889 & 451 \\
    			6 & 64 & 203401 & -469 \\
    			7 & 128 & 219961 & -4 \\
    			8 & 256 & 16 & 16 \\
    			9 & 512 & 256 & 256 \\
    			10 & 1024 & 65536 & 83 \\
    			\bottomrule
    		\end{tabular}
        \end{table}
        \item Finally we have: 
        \begin{align*}
            2^{1072} &= 2^{1024 + 32 + 16} = 2^{1024}\cdot2^{32}\cdot2^{16} \\
            2^{1024}\cdot2^{32}\cdot2^{16} &\equiv (83)(451)(83) \equiv (451)(451) \equiv -469 \,(mod\,1073) \\
            2^{1072} &\equiv -469 \,(mod\,1073)
        \end{align*}
    \end{enumerate}
    Since $2^{1072} \not\equiv 1 \,(mod\,1073)$ we can conclude that $1073$ is \textbf{not} a prime number. In fact, the prime factors of $1073$ are $29$ and $37$.
    
    \item When a node has itself as a parent, e.g. $\pi(v) = v$, it means that at that point in the algorithm the node $v$ is the root of the component that contains it. Moreover, the size of the root is equal to the number of nodes in the component, i.e. $size(v) = 35$ means that the component where $v$ is the root has 35 nodes (including $v$).\\
    In the case where $\pi(v) = u \neq v$, $size(v) = 35$ means that at some point in the past, when $v$ was its own parent and therefore a root, its component had $35$ nodes. At this point, however, the actual number of nodes in $v$'s component (where $u$ is the root) is given by $size(u)$.\\
    During the course of Kruskal's algorithm, $\pi(w)$ can take at most two values for each vertex $w$. The intuition behind this is that $\pi(w)$ will only be updated if two conditions are met: 1) if $w$ is the root of its component (otherwise \texttt{sliding-down-the-banister} will just keep going past $w$) and 2) if $size(w) < size(v)$, where $v$ is the other root being consider in the iteration. If at any point both conditions are met, $\pi(w) \leftarrow v$ will be updated. Once this happen (if it happens) it will never happen again, because $w$ will no longer be a root and the first condition will never be satisfied.\\
    During the course of Kruskal's algorithm, $size(w)$ can take at most $V$ values, where $V$ is the number of vertices in the graph. It is clear that $size(w)$ cannot be larger than $V$ because it would mean that there are more than $V$ nodes in $w$'s component. To explore the extreme case, let us consider a graph $G$ with edges of the form $e_i = \{1, i\}$ (for $2 \leq i \leq V$) that satisfies $w(e_i) < w(e_{j})$ if $i < j$. In such a graph we will get an update of the form $size(1) \leftarrow 1 + size(1)$ for every edge, so $size(1)$ will take all the values from $1$ to $V$.

\end{enumerate}

\end{document}