\documentclass{article}

% if you need to pass options to natbib, use, e.g.:
% \PassOptionsToPackage{numbers, compress}{natbib}
% before loading nips_2017
%
% to avoid loading the natbib package, add option nonatbib:
% \usepackage[nonatbib]{nips_2017}

%\usepackage{main}

% to compile a camera-ready version, add the [final] option, e.g.:
\usepackage[final]{main}

\usepackage[utf8]{inputenc} % allow utf-8 input
\usepackage[T1]{fontenc}    % use 8-bit T1 fonts
\usepackage{hyperref}       % hyperlinks
\usepackage{url}            % simple URL typesetting
\usepackage{booktabs}       % professional-quality tables
\usepackage{amsfonts}       % blackboard math symbols
\usepackage{nicefrac}       % compact symbols for 1/2, etc.
\usepackage{microtype}      % microtypography
\usepackage{multicol}
\usepackage{graphicx}
\usepackage{amsmath}
\usepackage{bbm}
\usepackage{enumerate}
\usepackage[linguistics]{forest}
\usepackage{adjustbox}
\usepackage{bbm}
\usepackage{stmaryrd}
%\usepackage[margin=0.5in]{geometry}
%\DeclareMathOperator*{\argmax}{argmax}

\usepackage{listings}
\lstset{
  basicstyle=\ttfamily,
  mathescape
}
\usepackage{color}
 
\definecolor{codegreen}{rgb}{0,0.6,0}
\definecolor{codegray}{rgb}{0.5,0.5,0.5}
\definecolor{codepurple}{rgb}{0.58,0,0.82}
\definecolor{backcolour}{rgb}{0.95,0.95,0.92}
 
\lstdefinestyle{mystyle}{
    backgroundcolor=\color{backcolour},   
    commentstyle=\color{codegreen},
    keywordstyle=\color{magenta},
    numberstyle=\ttfamily\tiny\color{codegray},
    stringstyle=\color{codepurple},
    basicstyle=\ttfamily\small,
    columns=fullflexible,
    breakatwhitespace=false,         
    breaklines=true,                 
    captionpos=t,                    
    keepspaces=true,                 
    numbers=left,                    
    numbersep=5pt,                  
    showspaces=false,                
    showstringspaces=false,
    showtabs=false,                  
    tabsize=4,
}
 
\lstset{style=mystyle}

\title{Fundamental Algorithms - Spring 2018\\
       \Large Homework 7}
%\graphicspath{{images/}}
\setcitestyle{round, sort, numbers}

% The \author macro works with any number of authors. There are two
% commands used to separate the names and addresses of multiple
% authors: \And and \AND.
%
% Using \And between authors leaves it to LaTeX to determine where to
% break the lines. Using \AND forces a line break at that point. So,
% if LaTeX puts 3 of 4 authors names on the first line, and the last
% on the second line, try using \AND instead of \And before the third
% author name.

\author{
  Daniel Rivera Ruiz\\
  Department of Computer Science\\
  New York University\\
  \href{mailto:drr342@nyu.edu}{\texttt{drr342@nyu.edu}}\\
}

\begin{document}

\maketitle

% \cite{} - in-line citation author, year in parenthesis.
% \citep{} - all citation info in parenthesis.

%	\begin{figure}[ht]
%		\centering
%		\frame{
%            \includegraphics[width=1.0\linewidth]{tree.png}
%       }
%		\caption{Classification results for the sentence \textit{"There are slow and repetitive parts, but it has just enough spice to keep it                  interesting."} using the Stanford Sentiment Treebank. As can be seen, sentiment scores are available for each phrase.}
%		\label{tree}
%	\end{figure}

\begin{enumerate}[1.]
    
    \item Given $X = 10010101$ and $Y = 010110110$ we use the definitions from class to build the matrices $c$ and $B$:
    \[
    c(i, j)= 
    \begin{cases}
        0                          &, \text{ if } i = 0 \text{ or } j = 0\\
        c(i-1, j-1) + 1            &, \text{ if } x_i = y_j\\
        \max[c(i-1, j), c(i, j-1)] &, \text{ otherwise}
    \end{cases}
    \]
    \[
    B(i, j)= 
    \begin{cases}
        Stop          &, \text{ if } i = 0 \text{ or } j = 0\\
        \nwarrow   &, \text{ if } x_i = y_j\\
        \leftarrow &, \text{ if } x_i \neq y_j \text{ and } c(i, j-1) \geq c(i-1, j)\\
        \uparrow   &, \text{ otherwise}
    \end{cases}
    \]
    
    The matrix $c$ looks as follows:
	\begin{table}[ht]
		\centering
		\begin{tabular}{c|cccccccccc}
			\toprule
			$i_{\downarrow} / j_{\rightarrow}$ & 0 & 1 & 2 & 3 & 4 & 5 & 6 & 7 & 8 & 9 \\
			\midrule
			0 & 0 & 0 & 0 & 0 & 0 & 0 & 0 & 0 & 0 & 0 \\
			1 & 0 & 0 & 1 & 1 & 1 & 1 & 1 & 1 & 1 & 1 \\
			2 & 0 & 1 & 1 & 2 & 2 & 2 & 2 & 2 & 2 & 2 \\
			3 & 0 & 1 & 1 & 2 & 2 & 2 & 3 & 3 & 3 & 3 \\
			4 & 0 & 1 & 2 & 2 & 3 & 3 & 3 & 4 & 4 & 4 \\
			5 & 0 & 1 & 2 & 3 & 3 & 3 & 4 & 4 & 4 & 5 \\
			6 & 0 & 1 & 2 & 3 & 4 & 4 & 4 & 5 & 5 & 5 \\
			7 & 0 & 1 & 2 & 3 & 4 & 4 & 5 & 5 & 5 & 6 \\
			8 & 0 & 1 & 2 & 3 & 4 & 5 & 5 & 6 & 6 & 6 \\
			\bottomrule
		\end{tabular}
	\end{table}
    
    And the matrix $B$ as follows:
	\begin{table}[ht]
		\centering
		\begin{tabular}{c|cccccccccc}
			\toprule
			$i_{\downarrow} / j_{\rightarrow}$ & 0 & 1 & 2 & 3 & 4 & 5 & 6 & 7 & 8 & 9 \\
			\midrule
			0 & S & S & S & S & S & S & S & S & S & S \\
			1 & S & $\leftarrow$ & $\nwarrow$ & $\leftarrow$ & $\nwarrow$ & $\nwarrow$ & $\leftarrow$ & $\nwarrow$ & $\nwarrow$ & $\leftarrow$ \\
			2 & S & $\nwarrow$ & $\leftarrow$ & $\nwarrow$ & $\leftarrow$ & $\leftarrow$ & $\nwarrow$ & $\leftarrow$ & $\leftarrow$ & $\nwarrow$ \\
			3 & S & $\nwarrow$ & $\leftarrow$ & $\nwarrow$ & $\leftarrow$ & $\leftarrow$ & $\nwarrow$ & $\leftarrow$ & $\leftarrow$ & $\nwarrow$ \\
			4 & S & $\uparrow$ & $\nwarrow$ & $\leftarrow$ & $\nwarrow$ & $\nwarrow$ & $\leftarrow$ & $\nwarrow$ & $\nwarrow$ & $\leftarrow$ \\
			5 & S & $\nwarrow$ & $\uparrow$ & $\nwarrow$ & $\leftarrow$ & $\leftarrow$ & $\nwarrow$ & $\leftarrow$ & $\leftarrow$ & $\nwarrow$ \\
			6 & S & $\uparrow$ & $\nwarrow$ & $\uparrow$ & $\nwarrow$ & $\nwarrow$ & $\leftarrow$ & $\nwarrow$ & $\nwarrow$ & $\leftarrow$ \\
			7 & S & $\nwarrow$ & $\uparrow$ & $\nwarrow$ & $\uparrow$ & $\leftarrow$ & $\nwarrow$ & $\leftarrow$ & $\leftarrow$ & $\nwarrow$ \\
			8 & S & $\uparrow$ & $\nwarrow$ & $\uparrow$ & $\nwarrow$ & $\nwarrow$ & $\leftarrow$ & $\nwarrow$ & $\nwarrow$ & $\leftarrow$ \\
			\bottomrule
		\end{tabular}
	\end{table}
    
   Using $B$ we start from the bottom right corner and follow a path that takes us to an $S$, looking for the diagonal arrows $\nwarrow$, which are associated to the values present in the LCS. For the current problem we get the following path, where the underlined positions correspond to a $\nwarrow$:
    \begin{center}
        \emph{Path = B(8, 9) - \underline{B(8, 8)} - B(7, 7) - \underline{B(7, 6)} - \underline{B(6, 5)} - B(5, 4) - \underline{B(5, 3)} - \underline{B(4, 2)} - \underline{B(3, 1)}}\\
    \end{center}
    \vspace{0.25cm}
    Finally, printing the underlined values of the path in reverse order we get the LCS:
    \begin{equation*}
        \textit{LCS} = 010101
    \end{equation*}
    
    \item The following table shows all the parenthesizations for $ABCDE$ according to the recursive formula for $P(n)$ with $n = 5$:
	\begin{table}[ht]
		\centering
		\begin{tabular}{ccccc}
			\toprule
			$i$ & $P(i)$ & $P(n - i)$ & Total & Parenthesizations \\
			\midrule
			1 & 1 & 5 & 5 &  $A(B(C(DE)))$, $A(B((CD)E))$, $A((BC)(DE))$ \\
			  &   &   &   &  $A(((BC)D)E)$, $A((B(CD))E)$\\
			2 & 1 & 2 & 2 &  $(AB)(C(DE))$, $(AB)((CD)E)$\\
			3 & 2 & 1 & 2 &  $(A(BC))(DE)$, $((AB)C)(DE)$\\
			4 & 5 & 1 & 5 &  $(A(B(CD)))E$, $(A((BC)D))E$, $((AB)(CD))E$ \\
			  &   &   &   &  $(((AB)C)D)E$, $((A(BC))D)E$\\
			\midrule
			& & & \textbf{14} & \\
			\bottomrule
		\end{tabular}
	\end{table}

\lstset{showlines=true}

	\item 
	\begin{enumerate}[(a)]
	    \item The following algorithm will find all the values $INC[i]$ for $1 \leq i \leq n$ in $O(n^2)$ time:
\begin{lstlisting}

    INC[i] $\leftarrow$ 1
    for i = 2 to n :
        max $\leftarrow$ 0
        for j = 1 to i :
            if ($x_j$ < $x_i$ and INC[j] > max) :
                max $\leftarrow$ INC[j]
            end if
        end for
        INC[i] $\leftarrow$ max + 1
    end for
    
\end{lstlisting}

        \item The following algorithm will find $LIS$ and $DIS$ in $O(n)$ time, assuming that the values $INC[i]$ and $DEC[i]$ are given for all $1 \leq i \leq n$:
\begin{lstlisting}

    LIS $\leftarrow$ 1
    DIS $\leftarrow$ 1
    for i = 2 to n :
        if (INC[i] > LIS) :
            LIS $\leftarrow$ INC[i]
        end if
        if (DEC[i] > DIS) :
            DIS $\leftarrow$ DEC[i]
        end if
    end for
    
\end{lstlisting}

        \item Since all the $x_i$ are different, there are only two possibilities to consider:
        \begin{itemize}
            \item $x_i < x_j$. In this case we know that $INC[j]$ will at least be equals to $INC[i] + 1$ (in the extreme case where the longest increasing subsequence is formed by the LIS up to $x_i$ plus the $x_j$ element). Therefore, we will always have $INC[i] \neq INC[j]$. 
            \item $x_i > x_j$. In this case we know that $DEC[j]$ will at least be equals to $DEC[i] + 1$ (in the extreme case where the longest decreasing subsequence is formed by the DIS up to $x_i$ plus the $x_j$ element). Therefore, we will always have $DEC[i] \neq DEC[j]$.
        \end{itemize}
        
        \item Given the sequence of length $m = ab + 1$, we label each number $x_i$ in the sequence with the pair $(INC[i], DEC[i])$. According to the previous result from $3(c)$, each two numbers in the sequence are labeled with a different pair: if $i < j$ and $x_i \leq x_j$ then $INC[i] < INC[j]$, and on the other hand if $x_i \geq x_j$ then $DEC[i] < DEC[j]$. If we assume that all the $INC[i]$ values are at most $a$ and all the $DEC[i]$ values are at most $b$, then we only have $ab$ possible values to label the elements of the sequence. However, we know that the length $m$ of the sequence is $ab + 1$, so there must be an element $x_i$ for which $INC[i]$ is greater than $a$ or alternatively $DEC[i]$ is greater than $b$. In the first case the resulting $LIS$ will be of length at least $a + 1$, and in the second the $DIS$ will have length at least $b + 1$. ($Q.E.D.$)
        
	\end{enumerate}
	
	\item The following code written in \texttt{Java} and based on the algorithm proposed in the textbook will print the minimum cost for the matrix-chain product, as well as the optimal parenthesization associated to it:
	
\begin{lstlisting}[language=Java]

public class MatrixChainMultiplication {
	
	public static void printParens(int[][] s, int i, int j) {
		if (i == j)
			System.out.print("A" + i);
		else {
			System.out.print("(");
			printParens(s, i, s[i-1][j-2]);
			printParens(s, s[i-1][j-2] + 1, j);
			System.out.print(")");
		}
	}

	public static void main(String[] args) {
		int[] p = {5, 10, 3, 12, 5, 50, 6};
		int n = p.length - 1;
		int[][] m = new int[n][n];
		int[][] s = new int[n-1][n-1];
		
		for (int i = 1; i <= n; i++) {
			m[i-1][i-1] = 0;
		}
		
		for (int d = 2; d <= n; d++) {
			for (int i = 1; i <= n - d + 1; i++) {
				int j = i + d - 1;
				m[i-1][j-1] = Integer.MAX_VALUE;
				for (int k = i; k <= j - 1; k++) {
					int q = m[i-1][k-1] + m[k][j-1] + p[i-1] * p[k] * p[j];
					if (q < m[i-1][j-1]) {
						m[i-1][j-1] = q;
						s[i-1][j-2] = k;
					}
				}
			}
		}
		
		System.out.println("min cost = " + m[0][n-1]);
		printParens(s, 1, n);
		
	}
	
}

\end{lstlisting}

    After running the code, we get a minimum cost value of 2,010 and the following parenthesization for the six matrices: $((A_1A_2)((A_3A_4)(A_5A_6)))$.
    \newline

    \item
    \begin{enumerate}[(a)]
        \item 
        \begin{gather*}
            \log_2\left(\frac{4^n}{\sqrt{n}}\right) = \log_2\left(4^n\right) - \log_2\left(\sqrt{n}\right) = n\cdot\log_2(4) - \frac{1}{2}\log_2(n) = 2n - \frac{1}{2}\log_2(n)
        \end{gather*}
        If we look at the simplified expression, the dominant factor is $2n$ and therefore the asymptotic value is $O(n)$.
        \item 
        \begin{align*}
            5^{313340} &< 7^{271251}\\
            \ln\left(5^{313340}\right) &< \ln\left(7^{271251}\right)\\
            313340\ln\left(5\right) &< 271251\ln\left(7\right)\\
            313340 \cdot 1.6094 &< 271251 \cdot 1.9459\\
            504,301.2755 &< 527,830.0738
        \end{align*}
        \item
        \begin{gather*}
            n^2\log_2\left(n^2\right) = 2n^2\log_2\left(n\right)\\
            \log_2^2\left(n^3\right) = \left[3\log_2\left(n\right)\right]^2 = 9\log_2^2\left(n\right)
        \end{gather*}
        \item 
        \begin{align*}
            e^{-\frac{x^2}{2}} &= \frac{1}{n}\\
            -\frac{x^2}{2} &= \ln\left(\frac{1}{n}\right)\\
            x^2 &= (-2)(-\ln(n))\\
            x &= \sqrt{2\ln(n)} 
        \end{align*}
        \item 
        \begin{gather*}
            \log_n\left(2^n\right) = n\log_n(2)\\
            \log_n\left(n^2\right) = 2\log_n(n) = 2
        \end{gather*}
        \item 
        \begin{gather*}
            \log_2(n) = \frac{\log_3(n)}{\log_3(2)}
        \end{gather*}
        \item If we double $k$ times the value of $i$ we get $2^ki$. So we are looking for the smallest $k$ such that $2^ki \geq n$ and therefore $k \geq \log_2\left(\frac{n}{i}\right)$.
        \item 
        \begin{gather*}
            \log_2\left[n^ne^{-n}\sqrt{2\pi n}\right] = n\log_2(n) - n\log_2(e) + \frac{1}{2}\log_2(2\pi) + \frac{1}{2}\log_2(n)
        \end{gather*}
        We observe that the second term of the sum is in the order of $O(n)$ (since $\log_2(e)$ is a constant), the third term is constant so it is $O(1)$ and the last term is in the order of $O(\log_2(n))$. Therefore, the leading term of the expression will be the first one, whose asymptotic value is $O(n\log_2(n))$.\\
        The bracketed expression inside the logarithm is interesting because it corresponds to the approximation for $n!$ according to Stirling's formula:
        \begin{gather*}
            n! \sim n^ne^{-n}\sqrt{2\pi n}
        \end{gather*}
    \end{enumerate}

\end{enumerate}
\end{document}