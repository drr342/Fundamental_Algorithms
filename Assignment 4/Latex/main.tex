\documentclass{article}

% if you need to pass options to natbib, use, e.g.:
% \PassOptionsToPackage{numbers, compress}{natbib}
% before loading nips_2017
%
% to avoid loading the natbib package, add option nonatbib:
% \usepackage[nonatbib]{nips_2017}

%\usepackage{main}

% to compile a camera-ready version, add the [final] option, e.g.:
\usepackage[final]{main}

\usepackage[utf8]{inputenc} % allow utf-8 input
\usepackage[T1]{fontenc}    % use 8-bit T1 fonts
\usepackage{hyperref}       % hyperlinks
\usepackage{url}            % simple URL typesetting
\usepackage{booktabs}       % professional-quality tables
\usepackage{amsfonts}       % blackboard math symbols
\usepackage{nicefrac}       % compact symbols for 1/2, etc.
\usepackage{microtype}      % microtypography
\usepackage{multicol}
\usepackage{graphicx}
\usepackage{amsmath}
\usepackage{bbm}
\usepackage{enumerate}
\usepackage[linguistics]{forest}
\usepackage{adjustbox}
\usepackage{bbm}
\usepackage{fancyvrb}
%\usepackage[margin=0.5in]{geometry}
%\DeclareMathOperator*{\argmax}{argmax}

\usepackage{listings}
\usepackage{color}
 
\definecolor{codegreen}{rgb}{0,0.6,0}
\definecolor{codegray}{rgb}{0.5,0.5,0.5}
\definecolor{codepurple}{rgb}{0.58,0,0.82}
\definecolor{backcolour}{rgb}{0.95,0.95,0.92}
 
\lstdefinestyle{mystyle}{
    backgroundcolor=\color{backcolour},   
    commentstyle=\color{codegreen},
    keywordstyle=\color{magenta},
    numberstyle=\ttfamily\tiny\color{codegray},
    stringstyle=\color{codepurple},
    basicstyle=\ttfamily\small,
    columns=fullflexible,
    breakatwhitespace=false,         
    breaklines=true,                 
    captionpos=t,                    
    keepspaces=true,                 
    numbers=left,                    
    numbersep=5pt,                  
    showspaces=false,                
    showstringspaces=false,
    showtabs=false,                  
    tabsize=4,
}
 
\lstset{style=mystyle}

\title{Fundamental Algorithms - Spring 2018\\
       \Large Homework 4}
%\graphicspath{{images/}}
\setcitestyle{round, sort, numbers}

% The \author macro works with any number of authors. There are two
% commands used to separate the names and addresses of multiple
% authors: \And and \AND.
%
% Using \And between authors leaves it to LaTeX to determine where to
% break the lines. Using \AND forces a line break at that point. So,
% if LaTeX puts 3 of 4 authors names on the first line, and the last
% on the second line, try using \AND instead of \And before the third
% author name.

\author{
  Daniel Rivera Ruiz\\
  Department of Computer Science\\
  New York University\\
  \href{mailto:drr342@nyu.edu}{\texttt{drr342@nyu.edu}}\\
}

\begin{document}

\maketitle

% \cite{} - in-line citation author, year in parenthesis.
% \citep{} - all citation info in parenthesis.

%	\begin{figure}[ht]
%		\centering
%		\frame{
%            \includegraphics[width=1.0\linewidth]{tree.png}
%       }
%		\caption{Classification results for the sentence \textit{"There are slow and repetitive parts, but it has just enough spice to keep it                  interesting."} using the Stanford Sentiment Treebank. As can be seen, sentiment scores are available for each phrase.}
%		\label{tree}
%	\end{figure}

\begin{enumerate}[1.]

    \item The following table shows the values of $T$ for the first iterations:
    	\begin{table}[h!t]
		\centering
		\begin{tabular}{ll}
			\toprule
			$n$ & $T(n)$\\
			\midrule
            1 & 1\\
            3 & $9T(1)+3^2=18=2\cdot3^2$\\
            9 & $9T(3)+9^2=243=3\cdot3^4$\\
            27 & $9T(9)+27^2=2,916=4\cdot3^6$\\
            81 & $9T(27)+81^2=32,805=5\cdot3^8$\\
            243 & $9T(81)+243^2=354,294=6\cdot3^{10}$\\
            $\ldots$ & $\ldots$\\
            $3^i$ & $9T(3^{i-1})+(3i)^2=(i+1)\cdot3^{2i}$\\
			\bottomrule
		\end{tabular}
	    \end{table}
	
	So in general we have the following system:
	\begin{gather*}
      \left\{
        \begin{array}{l}
    	    T(n) = (i+1)\cdot3^{2i}\\
    	    n = 3^i
        \end{array}
      \right.
    \end{gather*}
    Which gets us to the closed form
    \begin{equation*}
        T(n) = (\log_3(n)+1)\cdot n^2
    \end{equation*}
    Using the master theorem we observe that $\log_3(9)=2$ and $f(n)=n^2$, so we are in the just right overhead case and therefore
    \begin{equation*}
        T(n)=\Theta(n^2\log_3(n))
    \end{equation*}
    
    \item The following table shows the analysis for the three functions:
	\begin{table}[h!t]
		\centering
		\begin{tabular}{llllll}
			\toprule
			& $T(n)$ & $\log_b(a)$ & $f(n)$ & Type & Asymptotics\\
			\midrule
			a) & $6T(\frac{n}{2}) + n\sqrt{n}$ & 2.5850 & $\Theta(n^\frac{3}{2})$ & Low & $\Theta(n^{\textit{lg}6})$\\
			b) & $4T(\frac{n}{2}) + n^5$ & 2 & $\Theta(n^5)$ & High & $\Theta(n^5)$\\
			c) & $4T(\frac{n}{2}) + 7n^2 + 2n + 1$ & 2 & $\Theta(n^2)$ & Just right & $\Theta(n^2 \textit{lg}(n))$\\
			\bottomrule
		\end{tabular}
    \end{table}
    
    \item The recursion for the \texttt{Toom-3} algorithm is given by:
    \begin{equation*}
        T(n) = 5T\left(\frac{n}{3}\right) + O(n)
    \end{equation*}
    Using the master theorem (with low overhead since $1 < \log_35$) we get:
    \begin{equation*}
        T(n) = \Theta(n^{\log_35})
    \end{equation*}
    \newline
    To compare it with the Karatsuba algorithm, we observe that $log_23 \approx 1.5849$ and $log_35 \approx 1.4648$, so it is clear that, for large values of $n$, the \texttt{Toom-3} algorithm will be faster.
    
    \item
    \begin{enumerate}[(a)]
        \item We simply use the formula for the sum of squares up to $n$ and reduce from there:
        \begin{align*}
            S &= n^2 + (n+1)^2 + \ldots + (2n)^2\\
            S &= \frac{(2n)(2n+1)(4n+1)}{6} - \frac{(n-1)(n)(2n-1)}{6}\\
            S &= \frac{14 n^3 + 15 n^2 + n}{6}\\
            S &= \Theta(n^3)\\
            c_1 &= \frac{7}{3}, c_2 = \frac{15}{6}\\
        \end{align*}
        \item If we consider the whole series $S = \textit{lg}^2(1) + \textit{lg}^2(2) + \ldots + \textit{lg}^2(n)$, which has $n$ elements less or equal than $\textit{lg}^2(n)$, we conclude that $S < n\cdot \textit{lg}^2(n)$.\\
        Furthermore, if we take half the series $S = \textit{lg}^2(\frac{n}{2}) + \ldots + \textit{lg}^2(n)$, which has $\frac{n}{2}$ elements greater or equal than $\textit{lg}^2(\frac{n}{2})$, we conclude that $S > \frac{n}{2}\cdot \textit{lg}^2(\frac{n}{2})$.\\
        Putting this two inequalities together we get 
        \begin{equation*}
            \frac{n}{2}\cdot \textit{lg}^2\left(\frac{n}{2}\right) < S < n\cdot \textit{lg}^2(n)
        \end{equation*}
        Which for the asymptotic analysis yields $S = \Theta(n\cdot \textit{lg}^2(n))$.\\
        By simple observation of the previous inequalities, we can define the constants $c_1$ and $c_2$ as follows:
        \begin{equation*}
            c_1 = \frac{n}{2}, c_2 = n
        \end{equation*}
        \item We simply use the formula for the sum of cubes up to $n$ and reduce from there:
        \begin{align*}
            S &= 1^3 + 2^3 + \ldots + n^3\\
            S &= \left[\frac{n(n+1)}{2}\right]^2\\
            S &= \frac{n^4 + 2 n^3 + n^2}{4}\\
            S &= \Theta(n^4)\\
            c_1 &= \frac{1}{4}, c_2 = \frac{1}{3}\\
        \end{align*}
    \end{enumerate}
	    
    \item Algorithm to subtract two $n$-digit decimal numbers:\\
        \begin{Verbatim}[obeytabs]
SUBTRACT(A, B):
    INPUT = A[0...N], B[0...N]
    OUPUT = C[0...N]
    
    FOR I = 0 TO (N-1):
        IF A[I] < B[I]:
            A[I] += 10
            B[I+1]++
        C[I] = A[I] - B[I]
    C[N] = A[N] - B[N]
    return C
        \end{Verbatim}

In the worst-case scenario, we will perform four operations inside the \texttt{FOR} loop: one comparison, two additions and one subtraction. At the end there is one additional subtraction of the left-most digits. If we measure the time complexity of the algorithm as the number of operations performed we get $T(n) = 4n + 1$, which results in the asymptotic form $T(n) = \Theta(n)$.

\end{enumerate}
\end{document}